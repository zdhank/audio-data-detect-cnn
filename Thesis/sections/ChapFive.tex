% !TEX root=../main.tex
\chapter{Conclusion}
\renewcommand{\baselinestretch}{\mystretch}
\label{chap:Conclusion}
%\setlength{\parindent}{0pt}
\PARstart{T}{his} report proposed an automated audio detector based on the CNN architecture. The architecture of shallow network was implemented as baseline model. Mel-spectrogram is beneficial to represent audio signals due to its perceptual features.
It has been demonstrated that normalisation and balancing are essential before training, especially in data with large within-class variance. Moreover, we have investigated the improvement of several data processing methods. Performances of two noise reduction methods, namely spectral subtraction (SS) and MMSE-LSA estimators, were compared with visualisations of mel-spectrogram and model scores. The model with initial dataset using MMSE-LSA achieved outstanding scores in accuracy 84.1\% and F1 score of 78.9\%. However, the denoising method blurred the region of interest to some extent, leading to a relative higher generalisation error. The augmentation approach increased the variety and diversity of dataset, resulting in a decreased generalisation error. 

In order to further improve the performance, an deep neural network was proposed with the inspiration of VGGNet. The performance was limited for the baseline model due to the shallow architecture. Increasing network depth prominently improved the performance on classification. With the contributions of high-level features, the VGG-based model has the ability to learn hard positive examples. The optimal VGG-based model with augmentation data achieved expressively scores of accuracy in 85.05\% and F1 score in 83.32\%.  Even though training and predicting are time-cost work for VGG-based model, it is still significantly effective and efficient compared with manually labelling. Furthermore, hard negative mining was applied to train models a second time, which improve the accuracy and precision. Although there were a number of wrong predicted positions for all models, the prediction record was created in files for further manually check.
